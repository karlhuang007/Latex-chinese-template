\documentclass{ctexart}
\usepackage{cite}
\usepackage{amsmath}	% 使用数学公式
\usepackage{graphicx}	% 插入图片/PDF/EPS 等图像
\usepackage{subfigure}	% 使用子图像或者子表格
\usepackage{geometry}	% 设置页边距
\usepackage{fancyhdr}	% 设置页眉页脚
\usepackage{setspace}	% 设置行间距
\usepackage{hyperref}	% 让生成的文章目录有链接,点击时会自动跳转到该章节
\usepackage{url}
\usepackage{caption2}
% [geometry] 设置页边距
\geometry{top=2.6cm, bottom=2.6cm, left=2.45cm, right=2.45cm, headsep=0.4cm, foot=1.12cm}
% 设置行间距为 1.5 倍行距
\onehalfspacing
% 设置页眉页脚
\pagestyle{fancy}
%\lhead{左头标}
%\chead{\today}
%\rhead{152xxxxxxxx}
% \lfoot{}
% \cfoot{\thepage}
% \rfoot{}
\cfoot{\textbf{\pagemark}}
% Other classes are available too:
% \documentclass{ctexrep}
% \documentclass{ctexbook}
% \documentclass{ctexbeamer}

%% You can change the font if necessary.
% \setCJKmainfont{BabelStone Han}
% \setCJKsansfont{Noto Sans CJK SC}
\title{\centering\huge{\heiti Latex报告模板}}
\author{\small{\kaishu 你的名字}\\[2pt]
}

\begin{document}
\maketitle
% \title{}

\tableofcontents



\newpage
\section{Latex编译环境}
\begin{itemize}
\item 最方便在线编辑:https://www.overleaf.com/
\item 如果使用命令行方式进行编译,则需打开Windows 命令提示符或者*nix 的终端,在源代码所在的目录下输入:

pdflatex main.tex\\
或者\\
xelatex main.tex\\
\end{itemize}









\section{中国古代史}
\begin{itemize}
\item 原始社会
\item 奴隶社会
\item 封建社会的确立和初步发展——战国、秦、汉
\item 封建国家的分裂和民族大融合——三国、两晋、南北朝
\item 封建社会的繁荣——隋、唐
\item 民族融合的进一步加强和封建经济的继续发展——五代、辽、宋、夏、金、元
\item 统一的多民族国家的巩固和封建制度的逐渐衰落(鸦片战争以前)
\end{itemize}



\section{中国近代史}
\begin{enumerate}
\item 鸦片战争
\item 太平天国运动
\item 清朝后期资本主义的产生和民族危机的加深
\item 戊戌变法和义和团运动
\item 辛亥革命和清朝的灭亡
\item 中华民国初期北洋军阀的统治
\item 五四运动和中国共产党的创立
\item 第一次国内革命战争
\item 第二次国内革命战争
\item 抗日战争
\item 第三次国内革命战争
\end{enumerate}




\section{图片}

图片可以这么放:
\begin{figure}[h]
\centering
\includegraphics[width=1\textwidth]{image_part_002.jpg}
\caption{一张图片}
\label{fig:latex}
\end{figure}


\section{数学公式}
数学公式可以用Mathpix自动截取
\begin{equation}
\label{eq:sigmoid}
\centering{f(x) = \frac{1}{1 + e^{-x}}}.
\end{equation}

\section{表格}
表格可以这么写:

\begin{tabular}{|p{8em}|p{8em}|p{24em}|} %l(left)居左显示 r(right)居右显示 c居中显示 p{}单元格宽度固定为⟨width⟩,可自动折行
\hline 
参数名&类型&含义\\
\hline  
number & int & 计数\\
\hline 
time &double&时间\\
\hline 
\end{tabular}


\end{document}
